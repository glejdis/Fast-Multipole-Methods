\section{Implementation}
\label{implementation}

In this section we introduce a few points that we deem helpful for implementing the \gls{fmm}.

First, the particles might be distributed very unevenly inside the computational domain, e.g. in the case of colliding galaxies.
In these cases building the tree up to a fixed level, ignoring the particle distribution, results in a tree where either the majority of the particles is located in a small subset of the leaf nodes or the majority of leaf nodes is empty.
Both cases lead to non-linear complexity of the force calculation.
Therefore, it is important to subdivide boxes exactly if they contain more than a fixed number of particles.
The resulting tree consists of leaf boxes of varying sizes and is commonly called an adaptive tree \cite{short-course}.

Since adaptive trees contain leaf boxes of different sizes, neighboring leaf boxes might reside on different levels.
Stated differently, neighbors of leaf boxes might be inner nodes.
Remember that forces between neighboring leaf boxes must be calculated in a direct fashion.
In the case of adaptive trees, this is also true for forces acting between particles in the large leaf box and particles in all leaf boxes of the neighboring inner node.

Building the tree has complexity $\mathcal{O}(N \log N)$.
To ensure that this cost is negligible compared to the force calculations, it is important to build the tree only once at the start of the simulation.
However, in each time step every particle changes its position, and in consequence might move to a different leaf box.
Therefore, the tree has to be repaired.
This means moving particles to the correct node in the tree and minimizing the tree afterwards, i.e. deleting empty leaf boxes and merging leaf boxes which contain only a small number of particles \cite{griebel2007numerical}.

Lastly, obtaining the interaction lists can be quite painful \cite{SicchaSeminar}.
Our approach is to store links to the direct neighbors of each node in the tree data structure.
The corresponding code is easy to separate from the main algorithm and easy to test.
Thanks to the neighbor links it is straightforward to loop over all nodes in the interaction list of a child of a particular node, i.e. the interaction list of any node except the root node.
As the interaction list of the root node is empty, we can safely ignore it.
