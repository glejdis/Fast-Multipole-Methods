\section{The Algorithm}
\label{algorithm}

\newcommand\scaleFigures{0.8} % this variable defines how the figures for the steps are scaled, relatively to the chosen size.

This section focuses on explaining the \gls{fmm} algorithm by introducing the underlying data structure and giving a step by step introduction of its necessary operations.
The general idea behind the algorithm is to reduce the number of pairwise interactions by introducing approximations representing entire ``clusters'' of interactions.
By doing so, each of the $N$ target particles $p_j$ immediately interacts with only a few close particles, for all other ``distant'' interactions, the approximations are considered.

\subsection{A Tree of Boxes, Neighbors and Interaction Lists}

To achieve linear complexity, the \gls{fmm} utilizes a tree based structure, in our 2D case a quad-tree.
The root node of the tree corresponds to the domain $\Omega$.
From the root, the domain is recursively subdivided in four equal-sized nodes until every node contains less than $s$ particles.
Nodes that do not need further refinement are called the leafs or leaf nodes of the tree \cite{Martinsson2015}.

Given two distinct nodes $\sigma,\tau$, we define the following \cite{Martinsson2015}:
\begin{description}
  \item[The parent] of $\tau$ is the node on the next coarser level that contains $\tau$.
  \item[The children] of $\tau$ compose the set $\mathcal{L}_{\tau}^\text{child}$ of nodes with $\tau$ as their parent.
  \item[The neighbors] of $\tau$ compose the set $\mathcal{L}_{\tau}^\text{nei}$ of nodes on the same level that directly touch $\tau$.
  \item[The interaction list] of $\tau$ is the set $\mathcal{L}_{\tau}^\text{int}$ of all nodes $\sigma$ such that $\sigma$ and $\tau$ are on the same level, and do not touch each other, but their parents touch each other.
  \item[$\sigma$ is well-separated] from $\tau$ if they are on the same level and not neighbors.
\end{description}

\subsection{The Fast Multipole Method}

In the following, we elaborate on the main steps of the algorithm as described by Martinsson in \cite{Martinsson2015}, the mathematical details are given in \cref{operators}.

\begin{enumerate}
  \setcounter{enumi}{-1}

  \item \textbf{Construct the tree and all interaction lists.}
    \\
    Before calculating any forces, we need to build the tree data structure and determine the interaction lists, as described above.
    This has complexity $\mathcal{O}(N\log{N})$, however, it needs to be done only once during the initialization. Hence, for a sufficiently long simulation, this initial effort can be neglected.
    \\
    \cref{fig:tree} shows how the information of all source particles $p_i$ propagates to the target particles $p_j$.
    The differently colored arrows in this figure represent the different steps of the algorithm which are described below.
    \begin{figure}
  \centering
  \null\hfill
  \subfloat{%
    \begin{tikzpicture}[scale=1]
    \pgfmathsetseed{17}

      \coordinate (ydist) at (0,1);
      \coordinate (xdist) at (1,0);

      \coordinate (a) at (0,0);

        \coordinate (b) at ($ (a) - (ydist) - 2*(xdist) $);
        \draw (a) -- (b);

          \coordinate (d) at ($ (b) - (ydist) - (xdist) $);
          \draw (b) -- (d);

            \coordinate (h) at ($ (d) - (ydist) - 0.5*(xdist) $);
            \draw (d) -- (h);
            \coordinate (i) at ($ (d) - (ydist) + 0.5*(xdist) $);
            \draw (d) -- (i);

          \coordinate (e) at ($ (b) - (ydist) + (xdist) $);
          \draw (b) -- (e);

            \coordinate (j) at ($ (e) - (ydist) - 0.5*(xdist) $);
            \draw (e) -- (j);
            \coordinate (k) at ($ (e) - (ydist) + 0.5*(xdist) $);
            \draw (e) -- (k);

        \coordinate (c) at ($ (a) - (ydist) + 2*(xdist) $);
        \draw (a) -- (c);

          \coordinate (f) at ($ (c) - (ydist) - (xdist) $);
          \draw (c) -- (f);

            \coordinate (l) at ($ (f) - (ydist) - 0.5*(xdist) $);
            \draw (f) -- (l);
            \coordinate (m) at ($ (f) - (ydist) + 0.5*(xdist) $);
            \draw (f) -- (m);

          \coordinate (g) at ($ (c) - (ydist) + (xdist) $);
          \draw (c) -- (g);

            \coordinate (n) at ($ (g) - (ydist) - 0.5*(xdist) $);
            \draw (g) -- (n);
            \coordinate (o) at ($ (g) - (ydist) + 0.5*(xdist) $);
            \draw (g) -- (o);

      \foreach \x in { (a), (b), (c)
                     , (d), (e), (f)
                     , (g), (h), (i)
                     , (j), (k), (l)
                     , (m), (n), (o) } {
        \draw[fill] \x circle (2pt);
      }


      \coordinate (step) at (0, 0.8);
      \coordinate (startingPoint) at ($ (h) - (step) $);
      \foreach \i in {0,...,7}{
        \def\radius{0.2}
        \foreach \tmp in {1,...,50} {
          \coordinate (z) at (rnd*2*\radius-\radius, rnd*2*\radius-\radius);
          \draw[fill, black!40] ($ \i*(xdist)+(startingPoint)+(z)$) circle (0.5pt);
        }
      }
      \draw[fill, highlight] ($ 4*(xdist)+(startingPoint)+(z)$) circle (1pt)
        node[right] {\scriptsize\textcolor{black}{$p_j$}};
      \draw[fill, black!40] ($ 3*(xdist)+(startingPoint)+(z)$) circle (1pt)
        node[right] {\scriptsize\textcolor{black}{$p_i$}};

      \draw[fill, tfi] (l) circle (2pt);
      \node[draw=tfi, dashed, inner sep=8pt] at ($ 4*(xdist)+(startingPoint)$){};
      \draw[tfi, thick, ->] ($ (l)-(0,0.1) $)
        -- ($ 4*(xdist)+(startingPoint)+(0,0.4)$)
        node[midway, right] {\scriptsize\textcolor{black}{$\vct{T}_{\tau}^{\mathrm{tfi}}$}};

      \node[draw=dir, dashed, inner sep=8pt] at ($ 3*(xdist)+(startingPoint)$){};
      \draw[dir, thick, ->] ($ 3*(xdist)+(startingPoint)+(0.3,0)$) -- ($ 4*(xdist)+(startingPoint)-(0.3,0)$);
      \node[draw=dir, dashed, inner sep=8pt] at ($ 5*(xdist)+(startingPoint)$){};
      \draw[dir, thick, ->] ($ 5*(xdist)+(startingPoint)-(0.3,0)$) -- ($ 4*(xdist)+(startingPoint)+(0.3,0)$);

      \draw[ifo, thick, ->] ($ (j) + (0.1, +0.05) $)
        .. controls ($ (l) - (xdist) + (0,0.3) $)
        .. ($ (l) + (-0.1, +0.05) $);
      \draw[ifo, thick, ->] ($ (n) + (-0.1, +0.05) $)
        .. controls ($ (l) + (xdist) + (0,0.15) $)
        .. ($ (l) + (0.1, 0) $);
      \draw[ifo, thick, ->] ($ (o) + (-0.1, +0.05) $)
        .. controls ($ (l) + 1.5*(xdist) + (0,0.4) $)
        .. ($ (l) + (0.1, +0.1) $);
      \draw[ifo, thick, ->] ($ (d) + (0.1, +0.05) $)
        .. controls ($ (f) - 2*(xdist) + (0,0.5) $)
        .. ($ (f) + (-0.1, +0.1) $)
        node[midway, above] {\scriptsize\textcolor{black}{$\vct{T}_{\nu,\mu}^{\mathrm{ifo}}$}};

      \draw[ofs, thick, ->] ($ 0*(xdist)+(startingPoint)+(0,0.3)$)
        -- ($ (h) - (0,0.2) $)
        node[midway, right] {\scriptsize\textcolor{black}{$\vct{T}_{\sigma}^{\mathrm{ofs}}$}};

      \draw[ofs, thick, ->] ($ 1*(xdist)+(startingPoint)+(0,0.3)$)
        -- ($ (i) - (0,0.2) $);
      \draw[ofs, thick, ->] ($ 2*(xdist)+(startingPoint)+(0,0.3)$)
        -- ($ (j) - (0,0.2) $);
      \draw[ofs, thick, ->] ($ 6*(xdist)+(startingPoint)+(0,0.3)$)
        -- ($ (n) - (0,0.2) $);
      \draw[ofs, thick, ->] ($ 7*(xdist)+(startingPoint)+(0,0.3)$)
        -- ($ (o) - (0,0.2) $);

      \draw[ofo, thick, ->] ($ (h) + (-0.1,0.1) $)
        -- ($ (d) + (-0.15,-0.05) $);
      \draw[ofo, thick, ->] ($ (i) + (+0.1,0.1) $)
        -- ($ (d) + (+0.15,-0.05) $)
        node[midway, right] {\scriptsize\textcolor{black}{$\vct{T}_{\mu,\sigma}^{\mathrm{ofo}}$}};

      \draw[ifi, thick, ->] ($ (f) - (0.15,0.05) $)
        -- ($ (l) - (0.1,-0.1) $)
        node[midway, left] {\scriptsize\textcolor{black}{$\vct{T}_{\tau,\nu}^{\mathrm{ifi}}$}};

    \end{tikzpicture}
  }
  \hfill\subfloat{%
    \begin{tikzpicture}[scale=0.4]

      \draw[ofs, ->, ultra thick] (0, -0) -- (1, -0) node[right] (l1) {\scriptsize\textcolor{black}{step 1) outgoing from source expansion $\vct{T}_{\sigma}^{\mathrm{ofs}}$}} ;
      \draw[ofo, ->, ultra thick] (0, -1) -- (1, -1) node[right] (l2) {\scriptsize\textcolor{black}{step 2) outgoing from outgoing expansion $\vct{T}_{\mu,\sigma}^{\mathrm{ofo}}$}} ;
      \draw[ifo, ->, ultra thick] (0, -2) -- (1, -2) node[right] (l3) {\scriptsize\textcolor{black}{step 3) incoming from outgoing expansion $\vct{T}_{\nu,\mu}^{\mathrm{ifo}}$}} ;
      \draw[ifi, ->, ultra thick] (0, -3) -- (1, -3) node[right] (l4) {\scriptsize\textcolor{black}{step 4) incoming from incoming expansion $\vct{T}_{\tau,\nu}^{\mathrm{ifi}}$}} ;
      \draw[tfi, ->, ultra thick] (0, -4) -- (1, -4) node[right] (l5) {\scriptsize\textcolor{black}{step 5) target from incoming expansion $\vct{T}_{\tau}^{\mathrm{tfi}}$}} ;
      \draw[dir, ->, ultra thick] (0, -5) -- (1, -5) node[right] (l6) {\scriptsize\textcolor{black}{step 5) immediate computation}} ;
      \draw[fill, highlight] (0.5, -6) circle (3pt) node[right] (l7) at ($ (l6.west) + (0, -1)$){\scriptsize\textcolor{black}{target particles $p_j$}};
      \draw[fill, black!40 ] (0.5, -7) circle (3pt) node[right] (l8) at ($ (l7.west) + (0, -1)$){\scriptsize\textcolor{black}{source particles $p_i$}};

      \node (leftCorner) at ($ (l1.west) + (-1, 0) $) {};
      \node[draw, fit=(leftCorner) (l1) (l2) (l3) (l4) (l5) (l6) (l7) (l8)] {};
    \end{tikzpicture}
  }
  \hfill\null
  \caption{Flow of information through the tree (1D case)}
  \label{fig:tree}
\end{figure}

    \label{it:init}

  \item \textbf{Outgoing from source expansion $\vct{T}_{\sigma}^{\mathrm{ofs}}$}
    \\
%    Compute: $\hat{\vct{q}}^{\tau}=\vct{T}_{\tau}^{\mathrm{ofs}} \vct{q}\left(I_{\tau}\right)$
    Within each leaf node $\sigma$, the potentials of all source particles $p_i$ are approximated at the center $\vct c_\sigma$ of the node.
    These approximations are then summed up to a ``virtual pole'' $p_{\vct c_\sigma}$ as represented in \cref{fig:ofs:a,fig:ofs:b}.
    This representation of the potentials is called outgoing expansion and obtained by applying the operator $\vct{T}_{\sigma}^{\mathrm{ofs}}$ to the particles $p_i$.
    As will become clear in \cref{operators}, the outgoing expansion of $\sigma$ is valid only in nodes which are well-separated from $\sigma$.
    Since every particle contributes to exactly one outgoing expansion, the complexity of step~\ref{it:ofs} is $\mathcal{O}(N)$.

    \begin{figure}
      \centering
      \null\hfill%
      \subfloat[]{%
        \label{fig:ofs:a}
        \begin{tikzpicture}[scale=0.5*\scaleFigures]
          \pgfmathsetseed{42}
          \draw (0,0) rectangle (8,8);
          \foreach \x in {1,...,320} {
            \fill[black!40] (rnd*7.8+0.1, rnd*7.8+0.1) circle (2pt);
          }
          \fill[white] (2,2) rectangle (4,4);
          \foreach \x in {1,...,20} {
            \fill[black!40] (rnd*1.8+2.1, rnd*1.8+2.1) circle (2pt);
          }
          \draw[step=2] (0,0) grid (8,8);
          \draw[ultra thick, highlight] (3,3) circle (2);
          \node (sigma) at (3, 3) {\footnotesize{$\vct c_\sigma$}};
        \end{tikzpicture}
      }
      \hfill\subfloat[]{%
        \label{fig:ofs:b}
        \begin{tikzpicture}[scale=1*\scaleFigures]
          \pgfmathsetseed{42}
          \clip[draw] (3,3) circle (2);
          \draw (0,0) rectangle (8,8);
          \foreach \x in {1,...,320} {
            \fill[black!40] (rnd*7.8+0.1, rnd*7.8+0.1) circle (2pt);
          }
          \fill[white] (2,2) rectangle (4,4);
          \node (sigma) at (3,3) {};
          \foreach \x in {1,...,20} {
            \node (p) at (rnd*1.8+2.1, rnd*1.8+2.1) {};
            \fill[black!40] (p) circle (2pt);
            \draw[ofs, ->, thick] (p) -- (sigma);
          }
          \draw[step=2] (0,0) grid (8,8);
          \draw[ultra thick, highlight] (3,3) circle (2);
          \fill[ofs] (sigma) circle (3pt);
          \node[left] (sigma) at (3, 3) {\scriptsize{$p_{\vct c_\sigma}$}};
        \end{tikzpicture}
      }
      \hfill\subfloat[]{%
        \label{fig:ofs:c}
        \begin{tikzpicture}[scale=0.5*\scaleFigures]
          \pgfmathsetseed{42}
          \draw (0,0) rectangle (8,8);
          \draw[step=2] (0,0) grid (8,8);

          \foreach \x in {1,3,5,7} {
            \foreach \y in {1,3,5,7} {
              \fill[ofs] (\x, \y) circle (5pt);
          }}
          \node (sigma) at (3.6, 2.6) {\scriptsize{$p_{\vct c_\sigma}$}};
        \end{tikzpicture}
      }
      \hfill\null
      \caption{Step 1 of the \gls{fmm}}
      \label{fig:ofs}
    \end{figure}
    \label{it:ofs}

  \item \textbf{Outgoing from outgoing expansion $\vct{T}_{\mu,\sigma}^{\mathrm{ofo}}$}
    \\
%    Compute: $\hat{\vct{q}}^{\tau}=\sum_{\sigma \in \mathcal{L}_{\tau}^{\text {child }}} \vct T_{\tau, \sigma}^{\text {ofo }} \hat{\vct{q}}^{\sigma}$
    After approximating $p_{\vct c_\sigma}$ for all leaf nodes $\sigma$, the information must be distributed throughout the tree.
    To this end, all child nodes $\sigma$ pass their information to their parent node $\mu$ to determine its outgoing expansion.
    Just like before, the outgoing expansion is an approximation to the potential caused by the particles inside $\mu$, which is valid only in well-separated nodes.
    To avoid considering every particle again, the outgoing expansion of $\mu$ is obtained by translating the expansions of its children to the center of $\mu$ via the operator $\vct{T}_{\mu,\sigma}^{\mathrm{ofo}}$.
    By summing the translated expansions, we obtain the outgoing expansion of $\mu$ at the new ``virtual pole'' $p_{\vct c_\mu}$, compare \cref{fig:ofo}.
    For the approximated potential to be present at each node, this step is recursively repeated up to the root node.
    Steps \ref{it:ofs} and \ref{it:ofo} are therefore known as the upward-pass.
    As the total number of nodes in the tree is of order $N$, step~\ref{it:ofo} has complexity $\mathcal{O}(N)$.

    \begin{figure}
      \centering
      \null\hfill\subfloat[]{
      \label{fig:ofo:child}
        \begin{tikzpicture}[scale=0.5*\scaleFigures]
          \pgfmathsetseed{42}

          \fill[upward!30] (3,5) rectangle (2,4)
            node[left] (sigma) at (2.3,4.5) {\footnotesize\textcolor{black}{$\sigma$}};

          \draw[step=1] (0,0) grid (8,8);
          \draw[step=2, thick] (0,0) grid (8,8);

          \node (mu) at (3,5) {};
          \foreach[count=\i] \x in {(2.5,4.5), (3.5,4.5), (2.5,5.5), (3.5,5.5)} {
            \fill[ofs] \x circle (5pt);
            \draw[ofo, ->, thick] \x -- (mu);
          }
          \draw[ofo, pattern=north west lines, pattern color=ofo] (mu) circle (5pt);
          \draw[ofo, pattern=north east lines, pattern color=ofo] (mu) circle (5pt);
        \end{tikzpicture}
        }
      \hfill\subfloat[]{%
      \label{fig:ofo:parent}
        \begin{tikzpicture}[scale=0.5*\scaleFigures]
          \pgfmathsetseed{42}

          \fill[downward!30] (0,0) rectangle (8,2);
          \fill[downward!30] (6,0) rectangle (8,8);
          \fill[upward!30] (2,4) rectangle (4,6)
            node[left] (sigma) at (2.2,4.6) {\footnotesize\textcolor{black}{$\mu$}};

          \draw[step=2, thick] (0,0) grid (8,8);

          \node (mu) at (3,5) {};
          \foreach \x in {2.5,3.5} {
            \foreach \y in {4.5,5.5} {
              \node (sigma) at (\x, \y) {};
              \draw[ofs, pattern=north west lines, pattern color=ofs] (sigma) circle (5pt);
              \draw[ofs, pattern=north east lines, pattern color=ofs] (sigma) circle (5pt);
              \draw[ofo, ->, thick] (\x, \y) -- (mu);
          }}

          \fill[ofo] (mu) circle (5pt);
        \end{tikzpicture}
        }
      \hfill\null
      \caption{Step 2 of the \gls{fmm}. The yellow highlighted area is well-separated from $\mu$.}
      \label{fig:ofo}
    \end{figure}
    \label{it:ofo}

  \item \textbf{Incoming from outgoing expansion $\vct{T}_{\nu,\mu}^{\mathrm{ifo}}$}
 %    Compute: $\hat{\vct{u}}^{\tau}=\hat{\vct{u}}^{\tau}+\sum_{\sigma \in \mathcal{L}_{\tau}^{i n t}} \vct T_{\tau, \sigma}^{ifo} \hat{\vct{q}}^{\sigma}$ and $\hat{\vct{u}}^{\tau}=\hat{\vct{u}}^{\tau}+\vct T_{\tau, \sigma}^{\mathrm{ifi}} \hat{\vct{u}}^{\sigma}$
    \\
    After the upward-pass, every node $\mu$ in the tree contains approximations to the potentials caused by the particles located inside $\mu$.
    To calculate the interactions between particles, this information needs to be spread to the other nodes.
    For this, each node gathers the expansions from the nodes in its interaction list.
    More precisely, by applying the operator $\vct{T}_{\nu,\mu}^{\mathrm{ifo}}$ to the outgoing expansions of nodes in $\mathcal{L}_\nu^\text{int}$ and summing the results, the incoming expansion of $\nu$ is obtained.
    In contrast to the outgoing expansion, the incoming expansion approximates the potentials caused by particles located outside of $\nu$.
    As mentioned above, the outgoing expansions of the neighbors of $\nu$ are not valid inside $\nu$, hence, only well-separated nodes are considered.
    Likewise, interactions with nodes beyond the interaction list can be ignored since they are handled on the coarser levels (compare \cref{fig:ifo:parent,fig:ifo:child}).
    Step~\ref{it:ifo} also has a complexity of $\mathcal{O}(N)$ because a constant amount of work is performed for all $\mathcal{O}(N)$ nodes.
    \label{it:ifo}

  \item \textbf{Incoming from incoming expansion $\vct{T}_{\tau,\nu}^{\mathrm{ifi}}$}
    \\
    After step~\ref{it:ifo}, an incoming expansion is present at each node $\nu$ for each level.
    This only considers the nodes contained in the interaction list $\mathcal{L}_{\nu}^\text{int}$ of $\nu$.
    What is left, is to pass the potential information, gathered at each refinement level, down the tree up to the leaf nodes.
    Together with step~\ref{it:ifo} this is called the downward-pass.
    \\
    Complementary to $\vct{T}_{\mu,\sigma}^{\mathrm{ofo}}$ in step~\ref{it:ofo}, the potential information of the parent node $\nu$ is passed down to its children $\mathcal{L}_{\nu}^{\text{child}}$ and approximated at their respective centers $\vct c_\tau$ (compare \cref{fig:ifi}).
    Accordingly, step~\ref{it:ifi} consists of performing constant amount of work for each node of the tree, so its time complexity is $\mathcal{O}(N)$.

    Now it becomes clear why in step~\ref{it:ifo} we considered only nodes which are in the interaction list of $\nu$.
    As can be seen from \cref{fig:ifo:child}, at each refinement level there are some well-separated nodes, that were not well-separated on coarser levels, and therefore not considered yet.
    These missing approximations to the potentials from the nodes $\widetilde\mu$ are added to the incoming expansion of $\tau$ during the downward-pass.
    Consequently, on the finer level only the ``new'' well-separated nodes $\widetilde\mu$ which were not already considered on the coarser levels are elements of $\mathcal{L}_\tau^\text{int}$.

    \begin{figure}
      \centering
        \null\hfill
        \subfloat[]{%
        \label{fig:ifo:parent}
          \begin{tikzpicture}[scale=0.5*\scaleFigures]
            \pgfmathsetseed{42}

            \fill[downward!30] (2,4) rectangle (4,6);
            \fill[upward!30] (0,0) rectangle (8,2);
            \fill[upward!30] (6,0) rectangle (8,8);

            \draw[step=2, very thick] (0,0) grid (8,8);

            \node[left] (nu) at (3,5) {$\nu$};
            \node (nu) at (3,5) {};
            \foreach \x in {1,3,5,7} {
              \node (mu) at (\x,1) {};
              \fill[ofo] (mu) circle (5pt);
              \draw[ifo, ->, thick] (mu) -- (nu);
            }
            \foreach \y in {3,5,7} {
              \node (mu) at (7,\y) {};
              \fill[ofo] (mu) circle (5pt);
              \draw[ifo, ->, thick] (mu) -- (nu);
            }
            \node[below right] (mu) at (mu) {$\mu$};

            \fill[ifi] (nu) circle (5pt);
          \end{tikzpicture}
        }
        \hfill\subfloat[]{%
          \label{fig:ifi}
          \begin{tikzpicture}[scale=0.5*\scaleFigures]
            \pgfmathsetseed{42}
            \fill[downward!30] (2,4) rectangle (3,5);

            \draw[step=1            ] (0,0) grid (8,8);
            \draw[black!30, pattern=north west lines, pattern color=highlight!40] (0, 0) rectangle (8, 2);
            \draw[black!30, pattern=north west lines, pattern color=highlight!40] (6, 2) rectangle (8, 8);
            \draw[step=2, very thick] (0,0) grid (8,8);

            \node[below] (nu) at (3,5) {$\nu$};
            \node (nu) at (3,5) {};
            \foreach \tau in {(2.5, 4.5), (3.5, 4.5), (2.5, 5.5), (3.5, 5.5)}
            {
              \node (tau) at \tau {};
              \fill[ifi!50] \tau circle (5pt);
              \draw[ifi, ->] (nu.center) -- \tau;
            }
            \node (tau) at ($ (tau) + (0.2, -0.2) $) {$\tau$};
            \draw[ifi, pattern=north west lines, pattern color=ifi] (nu) circle (5pt);
          \end{tikzpicture}
        }
        \hfill\subfloat[]{%
          \label{fig:ifo:child}
          \begin{tikzpicture}[scale=0.5*\scaleFigures]
            \pgfmathsetseed{42}

            \fill[upward!30] (0, 6) rectangle (6, 8);
            \fill[upward!30] (0, 2) rectangle (1, 6);
            \fill[upward!30] (0, 2) rectangle (6, 3);
            \fill[upward!30] (4, 2) rectangle (6, 8);
            \fill[downward!30] (2,4) rectangle (3,5);

            \draw[black!30, pattern=north west lines, pattern color=highlight!40] (0, 0) rectangle (8, 2);
            \draw[black!30, pattern=north west lines, pattern color=highlight!40] (6, 2) rectangle (8, 8);

            \draw[step=1            ] (0,2) grid (6,8);
            \draw[step=2, very thick] (0,0) grid (8,8);

            \node (nu) at (2.5,4.5) {};

            \foreach \x in {0.5, 1.5, ..., 5.5}
            \foreach \y in {2.5, 6.5, 7.5} {
              \node (mu) at (\x,\y) {};
              \fill[ofo] (mu) circle (5pt);
              \draw[ifo, ->] (mu) -- (nu);
            }
            \foreach \x in {0.5, 4.5, 5.5}
            \foreach \y in {3.5, 4.5, 5.5} {
              \node (mu) at (\x,\y) {};
              \fill[ofo] (mu) circle (5pt);
              \draw[ifo, ->] (mu) -- (nu);
            }
            \node (mu) at ($ (mu) + (0.2, -0.2) $) {$\widetilde\mu$};

            \node[below, circle, inner sep=0.2pt, fill, color=white] (nu) at (2.5,4.5) {\textcolor{black}{$\tau$}};
            \fill[ifi] (2.5,4.5) circle (5pt);
          \end{tikzpicture}
        }
      \hfill\null
      \caption{Steps \ref{it:ifo} and \ref{it:ifi} of the \gls{fmm}. The interaction lists of the nodes $\nu$ and $\tau$ are highlighted in gray.}
      \label{fig:ifo}
    \end{figure}
    \label{it:ifi}

  \item \textbf{Target from incoming expansion $\vct{T}_{\tau}^{\mathrm{tfi}}$}
  %\item Compute: $\vct{u}\left(I_{\tau}\right)=\vct{T}_{\tau}^{\mathrm{tfi}}\hat{\vct{u}}^{\tau}+\vct{A}\left(I_{\tau}, I_{\tau}\right) \vct{q}\left(I_{\tau}\right)+\sum_{\sigma \in \mathcal{L}_{\tau}^{\mathrm{nei}}} \vct{A}\left(I_{\tau}, I_{\sigma}\right) \vct{q}\left(I_{\sigma}\right)$

    In the last step, the resulting potential approximations are applied to the target particles $p_j$.
    For each leaf node $\tau$, the incoming expansion is evaluated at the locations of the target particles $p_j$ and the resulting forces are calculated.
    This operation is denoted by $\vct{T}_{\tau}^{\mathrm{tfi}}$ and represented in \cref{fig:tfi:scope}.
    The complexity of this step is $\mathcal{O}(N)$ as it involves constant work for each particle.

    Until now, only the well-separated nodes on all refinement-levels were considered.
    However, as can be seen from \cref{fig:tfi:overview}, there are also not well-separated source particles $p_i$ that influence the target particles $p_j$, i.e. particles in the neighbors $\mathcal{L}_{\tau}^{\mathrm{nei}}$ of the current node $\tau$, as well as the node $\tau$ itself.
    These source particles are too close to the targets $p_j$ so that their potentials can not be approximated without introducing overly large errors.
    Therefore, they are handled in a direct manner using \cref{equ:force} as illustrated in \cref{fig:dir:scope}.
    Since the total number of direct source particles $p_i$ is bounded by the constant $9s$ for each target particle $p_j$, the complexity of the direct computation is $\mathcal{O}(N)$ as well.

    \begin{figure}
      \centering
      \null\hfill
      \subfloat[]{%
        \label{fig:tfi:overview}
        \begin{tikzpicture}[scale=0.5*\scaleFigures]
          \draw[black!30, pattern=north west lines, pattern color=highlight!40] (0, 0) rectangle (8, 8);
          \fill[white] (3, 3) rectangle (6, 6);
          \draw[step=1            ] (2,2) grid (6,6);
          \draw[step=2, very thick] (0,0) grid (8,8);

          \node (pj) at (rnd*0.8+4.1, rnd*0.8+4.1) {};
          \fill[tfi] (pj) circle (2pt);

          \foreach \x in {1,...,15} {%
            \node (pi) at (rnd*0.8+3.1, rnd*2.8+3.1) {};
            \fill[black!40] (pi) circle (2pt);
          }
          \foreach \x in {1,...,15} {%
            \fill[black!40] (rnd*0.8+5.1, rnd*2.8+3.1) circle (2pt);
          }
          \foreach \x in {1,...,5} {%
            \fill[black!40] (rnd*0.8+4.1, rnd*0.8+3.1) circle (2pt);
          }
          \foreach \x in {1,...,5} {%
            \fill[black!40] (rnd*0.8+4.1, rnd*0.8+5.1) circle (2pt);
          }

          \foreach \x in {1,...,4} {%
            \fill[tfi] (rnd*0.8+4.1, rnd*0.8+4.1) circle (2pt);
          }

          \fill[ifi] (4.5, 4.5) circle (5pt);
          \node[right] (nu) at (4.5, 4.5) {$\tau$};

          \draw[ultra thick, highlight] (4.5,4.5) circle (2.5);
          \draw[ultra thick, tfi] (4.5,4.5) circle (1);
        \end{tikzpicture}
      }
      \hfill\subfloat[]{%
        \label{fig:tfi:scope}
        \begin{tikzpicture}[scale=1.8*\scaleFigures]
          \clip[draw] (4.5,4.5) circle (1);
          \draw[black!30, pattern=north west lines, pattern color=highlight!40] (0, 0) rectangle (8, 8);
          \fill[white] (3, 3) rectangle (6, 6);
          \draw[step=1            ] (2,2) grid (6,6);
          \draw[step=2, very thick] (0,0) grid (8,8);

          \node (nu) at (4.5, 4.5) {};

          \node (pj) at (rnd*0.8+4.1, rnd*0.8+4.1) {};
          \fill[tfi] (pj) circle (1pt);
          \draw[tfi, ->] (nu) -- (pj);

          \foreach \x in {1,...,15} {
            \node (pi) at (rnd*0.8+3.1, rnd*2.8+3.1) {};
            \fill[black!40] (pi) circle (1pt);
          }
          \foreach \x in {1,...,15} {
            \node (pi) at (rnd*0.8+5.1, rnd*2.8+3.1) {};
            \fill[black!40] (pi) circle (1pt);
          }
          \foreach \x in {1,...,5} {
            \node (pi) at (rnd*0.8+4.1, rnd*0.8+3.1) {};
            \fill[black!40] (pi) circle (1pt);
          }
          \foreach \x in {1,...,5} {
            \node (pi) at (rnd*0.8+4.1, rnd*0.8+5.1) {};
            \fill[black!40] (pi) circle (1pt);
          }
          \node[below right] (pi) {$p_i$};
          \foreach \x in {1,...,4} {
            \node (pj) at (rnd*0.8+4.1, rnd*0.8+4.1) {};
            \fill[tfi] (pj) circle (1pt);
            \draw[tfi, ->] (nu) -- (pj);
          }
          \node[below left] (pj) at (pj) {\scriptsize{$p_j$}};

          \fill[ifi] (nu) circle (2.5pt);
          \node[right] (nu) at (4.5, 4.5) {$p_{\vct c_\tau}$}; \draw[ultra thick, tfi] (4.5,4.5) circle (1);
        \end{tikzpicture}
      }
      \hfill\subfloat[]{%
        \label{fig:dir:scope}
        \begin{tikzpicture}[scale=6/8*\scaleFigures]
          \clip[draw] (4.5,4.5) circle (2.5);
          \draw[black!30, pattern=north west lines, pattern color=highlight!40] (0, 0) rectangle (8, 8);
          \fill[white] (3, 3) rectangle (6, 6);
          \draw[step=1            ] (2,2) grid (6,6);
          \draw[step=2, very thick] (0,0) grid (8,8);

          \node (pj) at (rnd*0.8+4.1, rnd*0.8+4.1) {};

          \foreach \x in {1,...,15} {
            \node (pi) at (rnd*0.8+3.1, rnd*2.8+3.1) {};
            \fill[black!40] (pi) circle (2pt);
            \draw[dir, ->] (pi) -- (pj);
          }
          \node[below left] (tmp) at (pi) {\textcolor{black}{\scriptsize{$\widetilde{p}_i$}}};
          \foreach \x in {1,...,15} {
            \node (pi) at (rnd*0.8+5.1, rnd*2.8+3.1) {};
            \fill[black!40] (pi) circle (2pt);
            \draw[dir, ->] (pi) -- (pj);
          }
          \foreach \x in {1,...,5} {
            \node (pi) at (rnd*0.8+4.1, rnd*0.8+3.1) {};
            \fill[black!40] (pi) circle (2pt);
            \draw[dir, ->] (pi) -- (pj);
          }
          \foreach \x in {1,...,5} {
            \node (pi) at (rnd*0.8+4.1, rnd*0.8+5.1) {};
            \fill[black!40] (pi) circle (2pt);
            \draw[dir, ->] (pi) -- (pj);
          }
          \foreach \x in {1,...,4} {
            \node (pi) at (rnd*0.8+4.1, rnd*0.8+4.1) {};
            \fill[black!40] (pi) circle (2pt);
            \draw[dir, ->] (pi) -- (pj);
          }

          \node[below, circle, inner sep=0.2pt, fill, color=white] (tmp) at ($ (pj) + (-0.1, 0.1) $) {\textcolor{black}{\scriptsize{$p_j$}}};
          \fill[dir] (pj) circle (2pt);
          \draw[ultra thick, highlight] (4.5,4.5) circle (2.5);
        \end{tikzpicture}
      }
      \hfill\null
      \caption{Step \ref{it:tfi} of the \gls{fmm}}
      \label{fig:tfi}
    \end{figure}
    \label{it:tfi}
\end{enumerate}

As discussed, all steps of the \gls{fmm} have linear time complexity so the runtime of the \gls{fmm} is in $\mathcal{O}(N)$.
